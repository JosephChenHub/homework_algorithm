\section{special inversion counting}
\textbf{Question:} \\
Recall the problem of finding the number of inversions. As in the course, we are given a
sequence of n numbers $a_1,\cdots,a_n$, which we assume are all distinct, and we difine an inversion
to be a pair $i < j$ such that $a_i > a_j$.
We motivated the problem of counting inversions as a good measure of how different two
orderings are. However, one might feel that this measure is too sensitive. Let’s call a pair \textit{a significant inversion} if $i < j$ and $a_i > 3a_j$. Given an $O(nlogn)$ algorithm to count the number of significant inversions between two orderings. \\
\textbf{Solution:} 
\begin{enumerate}[a.)]
	\item \textbf{Algorithm description:} \\
	Unlike the trivial counting inversion problem, there is a point that how to count the significant inversion, it must be done with $O(n)$.
	pseudo-code:
	\begin{algorithm}[H]
		\caption{counting the special inversion}
		\begin{algorithmic}[1]
			\Require An unsorted sequence $a_1, a_2,\dots , a_n$	
			\Ensure  the inversion number
			\Function {merge}{$L = \{A_1,A_2,\cdots,A_m\}, R = \{B_1,B_2,\cdots, B_n\}$}
				\State $inv\_num = 0$, $i = 1$, $j = 1$
				\While {$i < m$ and $j < n$}
					\If {$A_i > 3*B_j$ 	}
						\State $inv\_num = inv\_num + m+1 - i$
						\State $j = j+1$
					\Else
						\State $i = i +1 $
					\EndIf 
				\EndWhile
				\State $C = \{\}$, $i = 1$, $j=1$
				\For {$k = 1$ to $m+n$}
					\If {$A_i <= B_j$}
						\State $append(C,{A_i})$
					\Else
						\State $append(C,{B_j})$
					\EndIf 
				\EndFor 
				
				\State \Return ($inv\_num$, $C$)
			\EndFunction 
			
			\Function {merge-count} {$A = \{a_1,a_2,\cdots, a_n\}$}
				   \If{$n=1$}
						\State\Return ($0$, $A$)
					\EndIf
				\State split the sequence into two halves $L$, $R$ 
				\State ($r_L$ , $L$) = \Call{merge-count}{$L$}
				\State ($r_R$ , $R$) = \Call{merge-count}{$R$}
				\State ($r_{LR}$, $L'$) = \Call{merge}{$L$, $R$}
				\State \Return ($r_L + r_R +r_{LR}$, $L'$)
			\EndFunction 
		\end{algorithmic}	
	\end{algorithm}
	\item \textbf{correctness} 
		\begin{itemize}
			\item correctness of ``merge'': since $L$ , $R$ are sorted sequence in ascending order, we scan these two lists. If $A_i > 3B_j$ then $A_i, A_{i+1} ,\cdots A_m > 3B_j$ , there is no need to continue scanning $L$ for $B_j$, then move to $B_{j+1}$, as $B_{j+1} > B_{j}, \forall k <i, A_{k} \leq 3B_j$, thus $A_k \leq 3B_{j+1}$, we just need to scan $L$ from last end point $A_i$ , once we reach each end point of $L$,$R$, we get the inversion number of two sorted lists
			\item  for an unsorted sequence $A$	, we can split it into two halves $L$, $R$,
			each part can be reduced into two halves iteratively , then we combine the results of two halves to get the final result.
			
			
		\end{itemize}
	
	\item \textbf{time complexity} \\	
	``merge'' manipulation costs $O(n)$, original problem is divided into two subproblems, thus
	\[
		T(n) = 2T(n/2) + O(n)
	\]
	$T(n) = O(n\log n)$
	
	
	
	
\end{enumerate}